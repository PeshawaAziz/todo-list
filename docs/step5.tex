\documentclass{article}
\usepackage{xepersian}
\usepackage{listings}
\usepackage{xcolor}
\usepackage{graphicx}
\usepackage{caption}
\usepackage[margin=3.6cm]{geometry}

\lstset{
    tabsize = 4,
    showstringspaces = false,
    commentstyle = \color{green},
    keywordstyle = \color{blue},
    stringstyle = \color{red},
    rulecolor = \color{black},
    basicstyle = \small \ttfamily,
    breaklines = true,
}

\settextfont{XB Yas}

\title{\textbf{گزارش‌کار پروژۀ عید - برنامه‌سازی پیشرفته (جاوا)}\vspace{1cm}\\پاسخ سوالات قدم پنجم}
\author{\textbf{پیشوا آزیز - \lr{40313003}}}
\date{}

\begin{document}

\maketitle

\vspace{2cm}

\section*{پاسخ سوال ۱}
در \lr{Java}، \lr{\lstinline{enum}} نوع خاصی از کلاس است که گروهی از ثابت‌ها را نگه‌داری می‌کند. برای تعریف \lr{\lstinline{enum}} از همین کلیدواژه استفاده می‌شود و ثابت‌ها با \lr{\lstinline{comma}} جدا می‌شوند. مثال:

\begin{latin}
\begin{lstlisting}[language=Java]
enum CarType {
    SEDAN,
    COUPE,
    SUV
}
\end{lstlisting}
\end{latin}

\section*{پاسخ سوال ۲}
یکی از دلایل وجود \lr{getter} و \lr{setter} در کلاس‌های \lr{\lstinline{Task}} و \lr{\lstinline{Step}} و قرار دادن شرط‌های بررسی مقادیر ورودی در داخل آن‌ها می‌تواند به‌دلیل صحت‌سنجی و امنیت مقادیر گرفته‌شده باشد. اما با وجود \lr{\lstinline{Validator}}ها، شاید لزومی به داشتن \lr{getter} و \lr{setter} نباشد؛ اما امکان دارد مزایایی هم داشته باشد.

\section*{پاسخ سوال ۳}
وجود فیلدی با محتوای لیستی از \lr{\lstinline{Step}}ها در هر \lr{\lstinline{Task}}، می‌تواند علاوه‌بر افزایش پیچیدگی کد و ناخوانایی آن، باعث وجود \lr{\lstinline{Task}}هایی با مقدار فضای اشغالی متفاوت در حافظه شود. همچنین برای پیاده‌سازی \lr{Deep copy} باید منطق اضافی برای آن نیز قرار داد که به نوبۀ خود باعث افزایش پیچیدگی کد می‌شود.

\section*{پاسخ سوال ۴}
به‌طور کلی، جداسازی بخش‌های مختلف برنامه بر اساس مسئولیت آن‌ها، مانند تفکیک بخش مدیریت \lr{I/O} از منطق برنامه، دارای اهمیت بسیار و پیروی از آن باعث تولید کدی با کیفیت بهتر، خوانایی بیشتر، نگه‌داری و گسترش و توسعه‌پذیری راحت‌تر در آینده می‌شود.

در صورتی که در آینده بخواهیم رابط کاربری برنامه را از محیط متنی (\lr{CLI}) به یک رابط گرافیکی (\lr{GUI}) تغییر دهیم، اگر منطق اصلی برنامه از ابتدا در کلاس جداگانه‌ای طراحی شده باشد، تنها لازم است نحوه‌ی دریافت ورودی و نمایش خروجی را تغییر دهیم، بدون آن‌که نیازی به بازنویسی منطق برنامه باشد. این امر باعث صرفه‌جویی در زمان توسعه و کاهش خطاها و باگ‌های احتمالی خواهد شد.

\end{document}